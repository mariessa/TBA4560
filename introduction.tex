
\externaldocument{acronyms.tex}
\chapter{Introduction}


In this ever-changing world we are more dependant on up-to-date geospatial data, and without frequent updating of map data the service databases can fast become out-of-date. Traditionally the management system of the basic map data (\textit{FKB}) in Norway was a local production at the municipalities, where the updated data was sent to the central data store once or twice a year. When occasions for updating are rare this may cause great data quality problems \citep{Lehto2015, Peng2005}. 

For about ten years the Norwegian cadastre, \textit{Matrikkelen}, has had a well functioning centralised production system \citep{Falkanger2017} - will this work for other geospatial systems in Norway as well? 
%Traditionally the management system of the basic map data in Norway was a local production at the municipalities - where the updated map data was sent to the central data store once or twice a year - but 
Today the new implementation of the the management system of the FKB map data in Norway follows a system where the municipalities directly update the map data to the data store managed by the Norwegian Mapping Authority, \textit{Kartverket}. Are there only advantages to this change, or are there some challenges by updating geospatial data directly in to a central data base?
In this paper we will try to answer this question, and the main objectives are:

\begin{enumerate}
\item To give a state of the art description of the management systems of the cadastre and FKB data.
\item To give an overview of techniques for transferring geospatial information, and see whether they are platform independent or not.
\item To discuss the pros and cons of having a centralised management system of geosaptial data.
\end{enumerate}



The paper is structured as follows: Initially there is a theoretical chapter presenting the two main registers of geospatial information in Norway. The next chapter is presenting techniques for updating and synchronising the main registers, as well as having a brief discussion on generic versus platform dependent APIs. 
%This is followed by a brief presentation of the managing systems in the neighbouring countries Sweden and Denmark. 
The following chapter discuss the benefits and disadvantages of updating data into a central data store. We close the paper with a conclusion on whether or not there are disadvantages to centralising the updating and management of the FKB data.

For an explanation of the central terms of this text, the reader may consult with the Appendix \ref{glossary}.

%All objectives shall be stated such that we, after having read the thesis, can see whether or not you have met the objective. ``To become familiar with \ldots'' is therefore not a suitable objective.
%The rest of the report is structured as follows. Chapter 2 gives an introduction to \ldots
%and will be further described in chapter \ref{ngis} kan brukes til senere
%As stated in Chapter \ref{SFKB}


%	\item "As a consequence, centralisation always seems to be the preferred solution for decision makers. More recently, however, new context conditions have led to this generally received wisdom being questioned (Marlow et al., 2013)" \cite{Eggimann2015}

%todo{sjekke over referanser}
%todo{Er det sections og subsections som bør slåes sammen, ikke stå for seg selv? og motsatt}
%todo er det de siste versjonene av figurer som brukes?
