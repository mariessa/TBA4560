\chapter{Directly Updating versus Local Updating}
The benefits of the new SFKB sytem are numerous, as compared to local storage; freshly updated data for the end-users of map data, and the changes will carry through to \textit{all} end-users, as opposed to local storage, which only applies to the users of that specific municipality (blir det riktig å anta at kommunene distribuerer fra sin lokale kopi mens de har de dataene?) \citep{Dontigney2017}. Another benefit is a more efficient and effective validation process, as the requirements are defined and validated one place and every change to the central store is done by the agency where the change actually occurred \citep{Kartverket2017e}. 

There are several authors that agrees on the befits, the Canadian geosynchronisation project was according to \cite{Reichardt2012} providing the most current and reliable data, consequently avoiding unnecessary versioning and keeping the duplication of data at a minimized level. 


Centralisation of data management within different scientific fields where sharing and analysing data is major tasks have been important for (...), and entering data directly into databases     In the paper \textit{A Generic framework for synchronidistributed data management in archaeological related disiplines} \cite{LOHRER2016558} discuss the advantages and disadvantages of centralised data management within the archaeological field

\subsubsection{Spatial Data Infrastuctures}

According to \cite{Peng2005} sharing data on file level is causing latency in update of the data, and it is difficult to deduce a generalised standardisation for all geodata, and reaching consensus might be difficult.  Spatial data infrastructure (SDI) 

Spatial data infrastructures seek to facilitate the access and integration of geospatial data coming from various sources. To achieve this objective, systems must be interoperable. \cite{giuliani2013} 
With the standard for exchanging geospatial data in Norway, SOSI (Samordnet Opplegg for Stedfestet Informasjon), 

One of the disadvantages of updating directly into a data store is that, as was stated earlier in this chapter, that it may be difficult to obtain a generalised data format and that validation thus will be difficult. As the Norwegian geospatial data have been standardised through the SOSI-standardisation for 30 years this will not be a problem (formuler annerledes). Validation problems wil mainly be because of humaly errors.    

%The SOSI format can be a quite touchy subject in the Norwegian geomatic's community. 

%The SOSI-format and the SOSI-standard 
%In Norway there is a standard for exchanging geospatial data: SOSI (Samordnet Opplegg for Stedfestet Informasjon). When mentioning SOSI it is hard not to 


%SOSI – Samordnet Opplegg for Stedfestet Informasjon. SOSI har hittil vært kjent som et format, men er mye mer. Standardene utarbeides som UML-modeller, og kan representeres ved ulike formater. SOSI er nå i en overgangsfase mellom SOSI prikkformat og SOSI GML-format. SOSI følger de internasjonale standardiseringsarbeidet i OGC og ISO/TC 211, samt INSPIRE.

%GML – Geographic Markup Language GML er et «rikt» format og står for Geographic Markup Language, og er basert på XML. Kartverket har lagt ned mye jobb i å gå fra SOSI prikkformat til GML-format. Det arbeides med mulighet for å levere FKB-data og arealplan-data på GML. Se Link: https://en.wikipedia.org/wiki/Geography_Markup_Language

%SOSI prikkformat: Det formatet som hittil har vært dominerende og godt kjent i Norge, men er tenkt erstattet av GML. SOSI prikkformat er et «rikt» format, men GML er «rikere». Det er vedtatt ikke å utvide det sær-norske SOSI prikkformatet og heller ta i bruk GML.





%Perhaps not as a seperate chaper, but as a part of the introduction?
\begin{itemize}
	\item definitions of centralized and decentralized system
	\item pros and cons
	\item Kartverket's web page of QA 
	\item check reference list: \url{https://en.wikipedia.org/wiki/Centralized_database} 
\end{itemize}

%"Centralized database storage, although it sometimes limits responsiveness to individual users, offers a number of key advantages for businesses."  \cite{Dontigney2017}

%"Centralized storage requires the business to invest heavily in the server technology, such as fault tolerance, but also allows it to cut overall costs. The maintenance to the central server proves less costly than maintenance to multiple computers, especially if the business operates in multiple locations. Centralized storage also reduces overall space requirements for data storage and processing.

%... improved reliability and update speed: Updates carried out on a database run on centralized storage carry through to all end-users, as opposed to local storage, which only applies to that computer. " \cite{Dontigney2017}



%Previous work - storing centrally

%\cite{Breslow2004}

%"Add to this the fact that file sizes and data storage requirements are increasing year after year, and the efficient sharing of files across distributed enterprises over the wide area network (WAN) has become a Herculean task.

%The problem is that although gigabytes of data can easily be shared over a local area network (LAN) using standard file server technology, they cannot so easily be shared across remote offices connected over the WAN. In truth, standard file server protocols provide unacceptably slow response times while opening and writing files over the WAN and this forces remote office IT managers to make some unappealing choices. IT managers and network users must either live with reduced productivity due to poor network performance at remote offices or they must use replication schemes that waste storage and inhibit global collaboration."




\begin{itemize}
	\item \cite{Breslow2004} WAN
	\item coherency
	\item consistency
\end{itemize}
%"Data coherency and data consistency are important properties of WAFS implementations, because they ensure that file updates are safe (cannot be written over) and available throughout the network of edge devices-crucial features for supporting engineering collaboration.

%Data coherency means that file updates (writes) from any one remote office are guaranteed never to conflict with updates from antrther remote office. Properly designed WAFS implementations guarantee this by maintaining a system of file leases. Leases are defined as a particular access privilege to a file from a remote office.

%...
%Data consistency implies that file updates made at one office are always available enterprise-wide, and well-architected WAFS system do this immediately after the update is made. Again, for collaboration, this is supremely important because remote designers want to be sure they are working on the most current version of any file, no matter where it was worked on last."


%Dette er utenfor scopet: 
%Taking a broader look at the concept of centralising, there are some fields that do not encourage centralising network infrastructures. Such as within the field of waste water management, electricity  and heater and water supply \cite{Eggimann2016}. 







\subsection{Previous work - generic API's}
This chapter will present papers concerning generic API's and framwork depending API's

%Bruk og argumenter for/mot bruk av generiske tjenester (WMS, WFS, WPS), kontra egne spesialiserte API'er (?). Plattformuavhengige API'er (WebService, REST) vs. plattformavhengigeAPI'er.

\cite{giuliani2013} wanted in their paper to benchmark and evaluate the quality of the web services WFS and WCS (Web Coverage Service). They conclude that amongst other things the OGC specifications (WFS, WCS) are providing interoperable access to data in an efficient and timely manner, and that the specifications are not convenient for transferring large volumes of data. 

----Kan jeg rerferere til samtale?-----
- zipped GML da gjør det ikke noe at det er store filer, smat større datamaskiner



%Because GML, SVG, WFS, and WMS are all international industry standards, the greatest strength of the proposed framework lies in the promised syntactic interoperability. \cite{YaoXiaobai2008Iimo}


%Several studies [8,13,21] have shown that projects that have adopted and implemented geospatial interoperability standards saved around 25 percent of their time, compared to those who rely on proprietary standards. These reports also showed that using geospatial interoperability standards lowers the transaction costs for sharing data and information. The fact that exchange of data and information is performed on standardized interfaces enhances flexibility and adaptability of projects over time. \cite{giuliani2013}
