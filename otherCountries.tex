\chapter{Equivalent systems in the other Scandinavian countries}
 As each country normally operates with separate file formats for the geospatial data \cite{Frenvik2017}
 Infrastructure for Spatial Information in the European Community (INSPIRE) is the European collaboration for a common standards for describing and sharing spatial data \cite{INSPIRE}. In the following sections there will be brief explanations of the geodata systems in Sweden and Denmark.


\section{Sweden}
The mapping authority in Sweden is Lantm\"{a}teriet. Lantm\"{a}teriet grant maps, aerial images and other geographical information about Sweden, together with central and local authorities. Students, scientist or employees of universities or other educational and cultural institutions can get the geodata provided by the Lantm\"{a}teriet for free. \todo{https://www.lantmateriet.se/en/Maps-and-geographic-information/} 

\subsection{The geodata cooperation, \textit{Geodatasamverkan}  evt geodataportalen} \todo{https://www.geodata.se/anvanda/geodatasamverkan/}

 
geodatas

\section{Denmark}
Geodatastyrelsen "Kartverket" http://gst.dk/om-os/
Datafordeleren http://sdfe.dk/hent-data/datafordeleren/
Grunndataprogrammet Sara Bjerre på sarbj@digst.dk
Matrikkelen http://gst.dk/matriklen/ejendomsdataprogrammet/matriklens-udvidelse/



\subsection{INSPIRE}
