%This is chapter 1
%%=========================================
\chapter{Introduction}
For nearly twenty years the Norwegian cadastre, \textit{Matrikkelen}, has had a centralised production system - will this work for other geospatial systems in Norway as well? Today the management system of the basic map data in Norway is being transferred from a local production at the municipalities - where the updated map data were sent to the central data store once or twice a year - to a system where the municipalities directly update the map data to the data store managed by the Norwegian Mapping Authority, \textit{Kartverket}. 

(Litt mer funfacts her)

%\begin{itemize}
%	\item Storing the MAtrikkel data centrally has worked for several years ..
%	\item cloud 
%	\item spatial data infrastructure  	\cite[p.~2616]{Warf2010}
%	\item According to several authors \cite{Peng2005} sharing data on file level is causing latency in update of the data. "A generalized standardization for all geospatial data is extremely difficult, and consensus is hard to achieve."
%	\item coherency and consistency \cite{Breslow2004}
%	\item motivation for this project
%	\item Sentrale begreper blir forklart i teksten, og henvise til Glossary.
%	\item The paper is organized as follows. 
%	\item "As a consequence, centralisation always seems to be the preferred solution for decision makers. More recently, however, new context conditions have led to this generally received wisdom being questioned (Marlow et al., 2013)" \cite{Eggimann2015}
%\end{itemize}


The organisation of the paper is as follows: Initially there are two theoretical chapters presenting the two main registers of geospatial information in Norway and the techniques for updating and synchronise to and from those. This is followed by a brief presentation of the managing systems in the neighbouring countries Sweden and Denmark. The following chapters discuss the benefits and disadvantages of updating data into a central data store, and includes as well a brief discussion on generic versus platform dependent API's. The paper ends with a conclusion whether there are some disadvantages to centralising the production of the basic map data.

%Today there is being implemented a map data management system called \textit{Sentral Felles Kartdatabase (SFKB)}, the central map registry. Using this system the agencies that manages the map data in Norway, e.g. the municipalities, will be able to update directly into a centralized database for geospatial data. With thi new system the users of the primary map data in Norway

%In this paper the two main registers of geospatial information in Norway will be presented and we question whether there are some disadvantages to centralising the production of the basic map data?

%Det denne prosjektoppgaven skal fokusere på fordeler og begrensninger ved å lagre data (FKB og matrikkeldata) sentralt, kontra lokal forvalting.




The main objectives of this project are
\begin{enumerate}
\item This is the first objective
\item This is the second objective
\item This is the third objective
\item More objectives
\end{enumerate}

%All objectives shall be stated such that we, after having read the thesis, can see whether or not you have met the objective. ``To become familiar with \ldots'' is therefore not a suitable objective.


The rest of the report is structured as follows. Chapter 2 gives an introduction to \ldots

\todo{sjekke over referanser}
\todo{Er det sections og subsections som bør slåes sammen, ikke stå for seg selv? og motsatt}
\todo er det de siste versjonene av figurer som brukes?



%and will be further described in chapter \ref{ngis} kan brukes til senere
%As stated in Chapter \ref{SFKB}

