\chapter{Conclusion}

Fordeler og ulemper med sentral lagring styrker og svakheter


\begin{itemize}
	\item Matrikkelen har fungert fint til nå
	\item oppsummere noe fra litterature review
	\item version-control issues as well as inefficient collaboration for storing local copies
	\item keeping service databases up-to-date in a multi-provider situation. %https://www.maanmittauslaitos.fi/sites/maanmittauslaitos.fi/files/attachments/2017/09/geoprocessing_2015_1_20_30079.pdf
	
\end{itemize}


%Fordel:
%---

%Ulempe:
%"A third key business objective was to address the increasing customer demands for improved application availability, not only in terms of failure recovery, but for the more important reduction of planned outage times. Today, there is less opportunity for planned systems shutdowns in the global economic environment. Here again, meeting this objective was key to the Parallel Sysplex cluster design." \cite{Nick1997}

%"Terry Walby, datacentre solutions director at Computacenter Services, warned that centralisation was putting huge pressure on power supplies and space in the datacentre. To mitigate the problems caused, Walby said server virtualisation was being explored by many firms, and storage virtualisation was also being taken more seriously." \cite{Tash2006}


%\cite{doi:10.1080/17538947.2017.1351583}   A paradigm-shift is needed, not only on the side of data providers, but also on the side of users who use large volumes of geospatial data. Data users have to shift from the traditional geospatial data workflow where large volumes of data have been downloaded and replicated onto local machines towards a workflow with integrated web service standards for data access and processing, that does not require time-consuming data download anymore. Data providers have to be more progressive towards offering server-based data access and processing in a standardised and interoperable way.


%Because GML, SVG, WFS, and WMS are all international industry standards, the greatest strength of the proposed framework lies in the promised syntactic interoperability. \cite{YaoXiaobai2008Iimo}
Avslutte med:
%The process of storing map data locally and later send those data to \textit{Kartverket} was perhaps unnecessarily time-consuming and inefficient. The (blablab diskusjon), the \textit{SFKB}-system will reduce operational cost by lessening the import, export, copying and control costs of the data.  This is in addition to improving service delivery for the public, providing it with fundamental and reliable geodata freshly updated at all times.  %https://www.kartverket.no/Prosjekter/Sentral-felles-kartdatabase/Ofte-stilte-sporsmal/ 

%As this paper has presented there are a few drawbacks of updating directly to a central data store. On the other hand, the benefits seem to be greater, and thus the SFKB-system have a bright future ahead; providing a greater national collaboration of the basic map data, where the end-users may rest assured of fresh data of high quality.


Further investigated 	Future work : This includes tracking latencies, bottlenecks, and errors that may negatively influence its overall quality. \cite{giuliani2013}

The process of storing map data locally and later send those data to \textit{Kartverket} was perhaps unnecessarily time-consuming and inefficient. As this paper has presented there are a few drawbacks of updating directly to a central data store. On the other hand, the benefits seem to be greater and thus the SFKB-system have a bright future ahead; the \textit{SFKB}-system will reduce operational cost by lessening the import, export, copying and control costs of the data, as \cite{Kartverket2017e} sums it up.  This is in addition to improving service delivery for the public, providing it with fundamental and reliable map data freshly updated at all times. 
