\chapter{Oppgaveteksten}
\textbf{TBA4560, Geomatikk, fordypningsprosjekt (15 stp)}\\
	%Trondheim, 2013-12-20\\[1pc]
	%Marie Senumstad Sagedal\\[1pc]
	Prosjektoppgåve for
	Marie Senumstad Sagedal\\

\noindent\textbf{Tittel:}\\
	Begrensninger ved sentral lagring av geografisk informasjon\\
%	\newline

\noindent\textbf{Bakgrunn:}\\
	I dag utvikles det et nytt sentralt kartdataforvaltningssystem kalt Sentral felles kartdatabase (SFKB). Ved hjelp av dette systemet skal de som forvalter kartdataene, f.eks. kommunene, kunne oppdatere dataene direkte i en sentral kartbase. På denne måten sikres brukerne av FKB-data tilgang til ferske og kvalitetssikre data til enhver tid. Tidligere ble de oppdaterte kartdataene kun sendt inn til Kartverket en til to ganger i året. Fra før av har vi i Norge en sentral matrikkeltjener og matrikkeldatabase, dette er et system som har fungert bra, og Sentral lagring vil ta i bruk tilsvarende teknikk \footnote{ Kartverket \textit{Innføring av Sentral FKB I kommunene} \url{http://www.kartverket.no/Prosjekter/Sentral-felleskartdatabase/sentral-lagring-av-fkb-data--innforing-i-kommunene/}}.

Det denne prosjektoppgaven skal fokusere på fordeler og begrensninger ved å 	lagre data (FKB og matrikkeldata) sentralt, kontra lokal forvalting.\\
%\newline

\noindent\textbf{Studenten skal gi:}
\begin{itemize}
	\item Informasjon om sentrale lagringssystemer av kart- og eiendomsdata i Norge (SFKB og Matrikkelen)
	\item Beskrivelse av teknikker for å synkronisere data (eksempelvis NGIS-API, Matrikkel-API og GeoSynkronisering)
	\item Bruk og argumenter for/mot bruk av generiske tjenester (WMS, WFS, WPS), kontra egne spesialiserte API'er (?). Plattformuavhengige API'er (WebService, REST) vs. Plattformavhengige API'er.
	\item Fordeler og ulemper ved sentral lagring.\\
\end{itemize}
%\newline

\noindent\textbf{Ekstern Vegleiar:}\\
	Lars Eggan, Norconsult Informasjonssystemer AS\\
%\newline

\noindent\textbf{NTNU Vegleiar:}\\
Terje Midtbø